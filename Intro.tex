Hiscox, a global specialist insurer, headquartered in Bermuda and listed on the London Stock Exchange which can be traced back to 1901 with roots in the Lloyd$'$s market. It is now a diversified international insurance group offering insurance products for nearly every need. One area growing rapidly in the company is Public Company Directors and Officers Liability (D\&O) insurance. This type of insurance protects directors of a publicly traded company against legal action from shareholders. \\
\begin{figure}[h]
  \centering
  \includegraphics[scale=0.4]{Process.png}
  \caption{Process of D\&O Claims}
\end{figure}
\setlength{\textfloatsep}{5pt}
\clearpage
\begin{figure}[h]
  \centering
  \includegraphics[scale=0.55]{Types.png}
  \caption{Types of Claim Policies}
\end{figure}
The first D\&O policy of UK can date back to 1930s when Lloyds of London underwrote it. Nowadays, D\&O are becoming part of risk management of a company. The coverage is for past, current and future directors and officers of a company as well as its subsidiaries. However, the coverage does not include criminal or fraudulent acts by directors for personal profit. Figure 1 shows what happened in practice.\\
\setlength{\textfloatsep}{5pt}
\begin{figure}[h]
  \centering
  \includegraphics[scale=0.35]{Excess.png}
  \caption{Excess Layer Structure}
\end{figure}
\setlength{\textfloatsep}{5pt}
\indent There are three different types of D\&O claims displayed in the figure above.The risk is always shared when larger size programs with limits over \$30m occur. For instance, from the left layer structure we can see that the lead insurers are always  more experienced and capable of handling wordings. They carry the most of layer up to 30m in this case which means they have to pay claims up to that amount. After reaching that limit, companies in the next layer will pay. Since the lead insurers hold the most of the risk, they receive a larger, non-proportional share of the premium. \\
\indent Understanding and predicting the outcomes of class action lawsuits brought by the US publicly traded companies is of great importance for insurance companies like Hiscox to diversify their business as well as control the risk as most of the claim costs lie in SCAs(Security Class Actions). However, the nature of the D\&O liability exposure, unlike auto insurance, leads to high difficulty to predict its amounts because of its high severity claims yet low frequency. In addition, following problems involve the collection of enough and reliable database to calibration and test of the final model. Nevertheless, the development of D\&O model has never stopped with lots of consulting and brokerage firms such as Advisen, Woodruff-Sawyer \& Co, NERA, Audit Integrity having built some sound models. Further developed from previous achievements is a hierarchical Bayesian predictive model built mainly by Blakeley McShane(2012) \hl{\cite{2}} from Northwestern University to forecast case results at the time the lawsuits are filed. To fully understand what the author concluded at the end, we need to know that the prediction based on a dataset consisted of around 1200 SCA lawsuits with associated cases between 1996 and 2005. Besides, the predictors the authors found that will positively affect the settle probability of a case include number of cases associated, S\&P500 return during the class period, individual plaintiff listed or not. The author also identified the variables that positively impact the cash settlement amount which are total number securities, market cap, class period length, company's return, number of Google hits, etc. On the other hand, the author employed a hierarchical Bayesian model for estimation of the federal circuit where the lawsuit filed as well as the defendant' industry group. The author can improve their thesis by adding more intelligible descriptive analytics like some business insights of D\&O market in the past. \\
\indent The model in this report adopted some similar features like S\&P500 return during class period and notoriety, yet more features are in the financial ratio field. Another contribution is a new time series forecasting model for monthly number of settle cases and monthly cash settlement amount. Furthermore, with more data coming in, this new analysis can contribute a more sounded model guaranteed by enough training, validation and testing dataset. Besides, the predictive modelling is more focused on selecting the model with the best performance among all kinds of features, algorithms and hyper-parameters.\\
\indent The deployed time series forecasting model can be used for early forecasting. Moreover, more accurate and automated identification of high versus low variance lawsuits will be achieved by prediction of the settlement probability and the settlement amount. This is useful for the affected D\&O insurers since they must have reliable means to assess and predict outcomes at the outset to try and set case reserves appropriately. In addition, just like what explained before, D\&O insurers whose coverage attaches only in the excess layers also need to have the ability to assess cases at the outset for determining the possibility that losses associated with any particular claim will penetrate their attachment point. The model could also provide a useful method for defendant companies as well as for their defence counsel in setting litigation strategy to get a precise understanding of the seriousness of the claim.\\
\indent The remaining part of the report will be organised as follows. Chapter 2 will display some significant summary statistics as well as market conditions and trends using different sources. Chapter 3 then will present a complete time series forecasting. In Chapter 4, the procedures of cleaning, preparing, feature generation, feature selection and model selection start from prepared dataset \textit{Total Segment Hiscox SCA Dataset} will be articulated. Finally, we will conclude the results, point the problems in the model, and provide some suggestions for future work in the last chapter.  