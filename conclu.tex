\squeezeup
\squeezeup
This paper has covered a wide range of knowledge on applying machine learning algorithms for prediction of securities class action lawsuits. A time series forecasting model was first introduced to predict the number of monthly settled cases. Then features selection techniques were used to reduce high potential variance in the model construction. In the end, both predictions on the likelihood of settlement and expected settlement amount were achieved with quite accurate fits to the data. 
Our settlement incidence prediction model revealed those factors with positive impact on the response include all kinds of allegation types, file prior class end date, file in Circuit 10(base Circuit 1), higher S\&P 500 return, open stock price, net worth, etc. Variables indicate a case is less likely to settle consist of longer time since IPO, higher market cap(without insider holding) or close stock price, file on account of FDA Approval Process(base: Sales Manipulation), and so on. 
We also found those predictors that are positively associated with the settlement amount contain market cap, class period length days, assets, S\&P 500 return, Google Hits, whether or not the case violated Insider Trading, GAAP, Clinical Trial Failure, or Stock Option Dating. Factors associated with lower settlement amount include violation of Misrepresentation, Non-Disclosure, company listed on Nasdaq or other exchanges compared to NYSE, file in Circuit 4(base: Circuit 1), etc. 
More interestingly, the cases predicted to be more probable to settle not necessarily settle with higher amounts. A good example in point is alleging misrepresentation and non-disclosure. This might be caused by the fact that the chance of pass a motion to dismiss will be higher despite the damage reward relatively low for a plaintiff. 
The main benefit from our model that an insurance company can take is the capability to identify those companies are relatively unlikely to settle and, once settled, will settle for a relatively smaller amount. Besides, defence company can take precautions while plaintiff company can leverage the opportunity to optimise their interest via the model. 
In closing, we provide some possible ways for improvements as well as more intuitive understanding of our prediction model. 
\begin{enumerate}
   \item Firstly, we can add more data from deleted ones to our dataset since more data always beats better algorithms.
   \item Next, some feature engineering methods such as Principle Component Analysis and Linear Discriminant Analysis can be utilised to reduce the dimension of the features, polynomial combinations of features can be added if the model is under fitted. 
   \item Afterwards, we can adjust all cash related features to one uniform measurement by inflation rate to remove the effect brought by cash depreciation. 
\item We can also map the industry return instead of S\&P 500 return for each sector as a more rational predictor. 
   \item For a better hyper parameters selection, we can apply grid search or smart search instead of just randomised search.
   \item Stacking model as a blending method can be utilised to combine logistic or linear regression results with other advanced machine learning algorithms. 
   \item Furthermore, we can uncover more underlying structures of the dataset by clustering algorithms like k-Means and then train a separate predictor on each cluster rather than train a single predictor on the entire dataset. Afterwards, we combine them to improve prediction accuracy. 
\end{enumerate}
Last but not the least, the dataset obviously is slightly biased due to the particularity of the response event which can be addressed by applying some statistical sample selection bias correction methods. 